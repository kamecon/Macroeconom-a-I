%\documentclass[blackandwhite]{beamer}
\documentclass{beamer}
\usepackage{beamerthemeshadow}
\usepackage[spanish]{babel}
\usepackage[latin1]{inputenc}
%\newtheorem{theorem}{Theorem}[section]
%\theoremstyle{definition}
%\newtheorem{definition}[theorem]{Definition}
\setbeamercolor{alerted text}{fg=blue!60!black}

\setbeamerfont{frametitle}{size=\normalsize}

\newcommand{\tbl}{\mbox{\hspace{0.5cm}}}

\newtheorem{proposition}[theorem]{Proposition}
%%%%%%%%%%%%%%%%%%%%%%%%%%%%%%
\usefonttheme[onlymath]{serif}  %%%get rid of helvetica in math!!!
%%~ altgr$&shift$
%%%%%%%%%%%%%%%%%%%%%%%%%%%%%%

\title[Balanza de Pagos] % (optional, use only with long paper titles)
{\textcolor{red}{Balanza de Pagos}}

%\subtitle{\ \\ {\textcolor{blue}{Work in Progress }}} %sacado de Luis

\author[Kamal Romero \ --- \ C.E.S Cardenal Cisneros]{{\large Kamal Romero} \\
%%\\ {\small {\em with the help of some friends}}
{\small C.E.S Cardenal Cisneros } }

\date{Curso 2017- 2018} \subject{Econom�a}

%%\date{{\ \vspace{-1.5cm}} ELSNIT, October 26-27th, 2007}

%%\beamertemplateshadingbackground{blue!25}{blue!0}
%%\definecolor{beameralert}{rgb}{0.2,0.4,0.8}
%%\definecolor{stabilo}{rgb}{1,1,0}
%%\definecolor{stabilo}{rgb}{0,0,1}

\newcommand{\tb}{\mbox{\hspace{-1.6cm}}}
\newcommand{\tba}{\mbox{\hspace{0.2cm}}}
\newcommand{\tbaL}{\mbox{\hspace{1.2cm}}}


% If you wish to uncover everything in a step-wise fashion, uncomment
% the following command:

%\beamerdefaultoverlayspecification{<+->}


\begin{document}

\begin{frame}
  \titlepage
\end{frame}

%\begin{frame}
%  \frametitle{Organizaci�n}
%  \tableofcontents
%  % You might wish to add the option [pausesections]
%\end{frame}


% Structuring a talk is a difficult task and the following structure
% may not be suitable. Here are some rules that apply for this
% solution:

% - Exactly two or three sections (other than the summary).
% - At *most* three subsections per section.
% - Talk about 30s to 2min per frame. So there should be between about
%   15 and 30 frames, all told.

% - A conference audience is likely to know very little of what you
%   are going to talk about. So *simplify*!
% - In a 20min talk, getting the main ideas across is hard
%   enough. Leave out details, even if it means being less precise than
%   you think necessary.
% - If you omit details that are vital to the proof/implementation,
%   just say so once. Everybody will be happy with that.


%\section{Our Goal}


\begin{frame}
   \frametitle{}

La \textbf{Balanza de Pagos} resume las transacciones  entre residentes y no residentes seg�n su categor�a:

\begin{columns}[T] % align columns
\begin{column}{.48\textwidth}
\color{red}\rule{\linewidth}{4pt}

Inversi�n
\end{column}%
\hfill%
\begin{column}{.48\textwidth}
\color{blue}\rule{\linewidth}{4pt}

Gasto
\end{column}%
\end{columns}

\begin{center}
\includegraphics<1>[height=5cm]{invgas.jpg}
\end{center}

\end{frame}



\begin{frame}
   \frametitle{Ejemplo}

Supongamos que existen solo 2 pa�ses en el mundo: USA y China

\begin{itemize}
\item Wal-Mart compra juguetes a China
\end{itemize}

\begin{center}
\includegraphics<1>[height=5cm]{walmart.jpg}
\end{center}



\end{frame}

\begin{frame}
   \frametitle{Ejemplo}

China ahora tiene moneda extranjera (US\$). �Qu� hace con ella?

\end{frame}

\begin{frame}
   \frametitle{}

China ahora tiene moneda extranjera (US\$). �Qu� hace con ella?

\begin{itemize}
\item Gasta en bienes y servicios de USA
\end{itemize}

\begin{center}
\includegraphics<1>[height=5cm]{salud.jpg}
\end{center}


\end{frame}

\begin{frame}
   \frametitle{Ejemplo}

China ahora tiene moneda extranjera (US\$). �Qu� hace con ella?

\begin{itemize}
\item O realiza inversiones en USA
\end{itemize}

\begin{center}
\includegraphics<1>[height=5cm]{invest.jpg}
\end{center}


\end{frame}

\begin{frame}
   \frametitle{Ejemplo}

China ahora tiene moneda extranjera (US\$). �Qu� hace con ella?

\begin{itemize}
\item Gran parte de esas inversiones son en deuda p�blica 
\end{itemize}

\begin{center}
\includegraphics<1>[height=6cm]{bills.jpg}
\end{center}


\end{frame}

\begin{frame}
   \frametitle{Ejemplo}

Y si despu�s de todo lo anterior, \textcolor{red}{�siguen sobrando divisas?}

\begin{itemize}
\item China acumula reservas. Divisas en el sistema financiero y el banco central
\end{itemize}

\begin{center}
\includegraphics<1>[height=6cm]{reserve.jpg}
\end{center}


\end{frame}



\begin{frame}
   \frametitle{La Identidad de la Balanza de Pagos}

\begin{center}
Cuenta corriente = ( - ) [ Cuenta de capital  + Variaci�n de reservas ]
\end{center}


\end{frame}

\begin{frame}
   \frametitle{La Identidad de la Balanza de Pagos}

\begin{center}
\includegraphics<1>[height=7cm]{BdP_diaz.pdf}
\end{center}


\end{frame}


\begin{frame}
   \frametitle{Datos}

   Cuenta corriente: \textbf{Espa�a}

\begin{center}
\includegraphics<1>[height=5.4cm]{cc.pdf}
\end{center}



\end{frame}

\begin{frame}
   \frametitle{Datos}
   
   Balanza de pagos : \textbf{Espa�a}

\begin{center}
\includegraphics<1>[height=5.4cm]{balance}
\end{center}


\end{frame}
%
%\begin{frame}
%   \frametitle{}
%
%
%\end{frame}
%
%\begin{frame}
%   \frametitle{}
%
%
%\end{frame}
%
%\begin{frame}
%   \frametitle{}
%
%
%\end{frame}
%
%\begin{frame}
%   \frametitle{}
%
%
%\end{frame}
%
%\begin{frame}
%   \frametitle{}
%
%
%\end{frame}

\frame{\titlepage}

% All of the following is optional and typically not needed.
\appendix
\section<presentation>*{\appendixname}
\subsection<presentation>*{For Further Reading}



\end{document}
